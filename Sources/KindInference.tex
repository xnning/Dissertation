%%%%%%%%%%%%%%%%%%%%%%%%%%%%%%%%%%%%%%%%%%%%%%%%%%%%%%%%%%%%%%%%%%%%%%%% 
\chapter{Kind Inference for Datatypes}
\label{chap:kindinference}
%%%%%%%%%%%%%%%%%%%%%%%%%%%%%%%%%%%%%%%%%%%%%%%%%%%%%%%%%%%%%%%%%%%%%%%% 

In recent years, languages like Haskell have seen a dramatic surge of new
features that significantly extends the expressive power of their type systems.
With these features, the challenge of \emph{kind inference} for datatype
declarations has presented itself and become a worthy research problem on its
own.

In this section, we apply promotion to kind inference for datatypes. Inspired by
previous research on type-inference, we offer declarative specifications for
what datatype declarations should be accepted, both for \hne and for a more
advanced system we call \tit, based on the extensions in modern Haskell,
including a limited form of dependent types. We believe these formulations to be
novel and without precedent, even for \hne. These specifications are
complemented with implementable algorithmic versions. We study \emph{soundness},
\emph{completeness} and the existence of \emph{principal kinds} in these
systems, proving the properties where they hold. This work can serve as a guide
both to language designers who wish to formalize their datatype declarations and
also to implementors keen to have principled inference of principal types.

\input{Gen/KindInference/introduction.lhstex}
\input{Gen/KindInference/haskell98.lhstex}
\input{Gen/KindInference/haskell98_algo.lhstex}
\input{Gen/KindInference/typeintype.lhstex}

%%% Local Variables:
%%% mode: latex
%%% TeX-master: "../Thesis"
%%% org-ref-default-bibliography: "../Thesis.bib"
%%% End: