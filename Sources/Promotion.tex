%%%%%%%%%%%%%%%%%%%%%%%%%%%%%%%%%%%%%%%%%%%%%%%%%%%%%%%%%%%%%%%%%%%%%%%% 
\chapter{Higher-Rank Type Inference Algorithm with Type Promotion}
\label{chap:Promotion}
%%%%%%%%%%%%%%%%%%%%%%%%%%%%%%%%%%%%%%%%%%%%%%%%%%%%%%%%%%%%%%%%%%%%%%%% 

Designing type inference algorithms is challenging. In particular, while a
declarative type system can \textit{guess} a type (e.g., the type of
\lstinline{x} in \lstinline{\x. e}) and can even be \textit{non-deterministic}
in the guessing (e.g., multiple possible types for \lstinline{x} in
\lstinline{\x. x}), an type inference algorithm needs to be deterministic, and
thus has to deal with many low-level details including \textit{unification}. At the
same time we expect an type inference algorithm to retain desirable properties
like \textit{inference of principal types}, as well as \textit{soundness} and
\textit{completeness} with respect to the declarative type system. An algorithm
is sound and complete, if it accepts and only accepts programs that are
well-typed in the declarative type system.

In this chapter, we focus on the design of type inference algorithms in the
presence of higher-rank polymorphism. Compared to type inference for simple
types, type inference for higher-rank polymorphism needs to further take care of
scoping and dependency issues between different kinds of variables. We propose a
strategy called \textit{promotion} that helps resolve the dependency of
unification variables in the framework of \textit{type inference in
  context}~\citep{gundry2010type}. To illustrate the key idea,
\Cref{sec:pr:unif} applies promotion to the unification algorithm for the simply
typed lambda calculus. \Cref{sec:pr:subtypin} then proposes \textit{polymorphic
  promotion} to deal with subtyping for higher-rank polymorphism, which leads to
an arguably simpler type inference algorithm for higher-rank polymorphism.
Finally, we briefly discuss how promotion can be further applied to other
advanced features like dependent types and gradual types in
\Cref{sec:pr:discussion}. This chapter also sets up the stage for
\Cref{chap:kindinference}, where promotion is used in a more complex setting.


\input{Gen/Promotion/introduction}
\input{Gen/Promotion/unification}
\input{Gen/Promotion/subtyping}
\input{Gen/Promotion/discussion}

%%% Local Variables:
%%% mode: latex
%%% TeX-master: "../Thesis"
%%% org-ref-default-bibliography: "../Thesis.bib"
%%% End:
