%%%%%%%%%%%%%%%%%%%%%%%%%%%%%%%%%%%%%%%%%%%%%%%%%%%%%%%%%%%%%%%%%%%%%%%% 
\chapter{Higher-Rank Type Inference with Type Promotion}
\label{chap:Promotion}
%%%%%%%%%%%%%%%%%%%%%%%%%%%%%%%%%%%%%%%%%%%%%%%%%%%%%%%%%%%%%%%%%%%%%%%% 

\cite{gundry2010type} proposed type inference in context as a general foundation
for unification/type inference algorithms. The key idea is based on the notion
of information increase. However, their semantic definition of information
increase is semantic makes it hard to prove metatheory formally. Following this
work, a more syntactic foundation for information increase is presented by
\cite{DK} (\Cref{sec:DK}) to deal with higher-rank polymorphism. However, the
DK approach produces duplication and cannot be easily generalized for more
complicated types.

In this section, we propose a strategy called \textit{promotion} that helps
resolve the dependency of existential variables in the framework of type
inference in context. \Cref{sec:pr:unif} shows that promotion works for the
unification algorithm for the simply typed lambda calculus.
\Cref{sec:pr:subtypin} proposes \textit{polymorphic promotion} to deal with
subtyping for higher-rank polymorphism. Finally, \Cref{sec:pr:discussion}
discusses how to promote dependent types and gradual types.


\input{Gen/Promotion/introduction}
\input{Gen/Promotion/unification}
\input{Gen/Promotion/subtyping}
\input{Gen/Promotion/discussion}

%%% Local Variables:
%%% mode: latex
%%% TeX-master: "../Thesis"
%%% org-ref-default-bibliography: "../Thesis.bib"
%%% End:
