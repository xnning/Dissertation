%%%%%%%%%%%%%%%%%%%%%%%%%%%%%%%%%%%%%%%%%%%%%%%%%%%%%%%%%%%%%%%%%%%%%%%% 
\chapter{Restoring the Dynamic Gradual Guarantee with Type Parameters}
\label{chap:Dynamic}
%%%%%%%%%%%%%%%%%%%%%%%%%%%%%%%%%%%%%%%%%%%%%%%%%%%%%%%%%%%%%%%%%%%%%%%% 

\ningning{This chapter turns out to be pretty short. I guess it's OK?}

In \Cref{sec:gradual:type:trans} we have seen an example where a single source expression could
produce two different target expressions with different runtime behaviors. As we
explained, this is due to the guessing nature of the declarative system, and,
from the (source) typing point of view, no guessed type is particularly better than 
any other. As a consequence, this breaks the dynamic gradual guarantee as discussed in \cref{sec:gradual:criteria}.

To alleviate this situation, we introduce \textit{static type parameters}, which
are placeholders for monotypes, and \textit{gradual type parameters}, which are
placeholders for monotypes that are consistent with the unknown type. The
concept of static type parameters and gradual type parameters in the context of
gradual typing was first introduced by \citet{garcia:principal}, and later
played a central role in the work of \citet{yuu2017poly}. In our type system,
type parameters mainly help capture the notion of \textit{representative
  translations}, and should not appear in a source program.
With them we are able to recast the dynamic gradual guarantee in terms
of representative translations, and to prove that every well-typed source expression
possesses at least one representative translation. With a
coherence conjecture regarding representative translations, the dynamic gradual
guarantee of our extended source language now can be reduced to that of \pbc,
which, at the time of writing, is still an open question. TODO: not open anymore?

\input{Gen/Dynamic/declarative}
\input{Gen/Dynamic/algorithm}

%%% Local Variables:
%%% mode: latex
%%% TeX-master: "../Thesis"
%%% org-ref-default-bibliography: "../Thesis.bib"
%%% End: