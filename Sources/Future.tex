%%%%%%%%%%%%%%%%%%%%%%%%%%%%%%%%%%%%%%%%%%%%%%%%%%%%%%%%%%%%%%%%%%%%%%%% 
\chapter{Summary and Future Directions}
\label{chap:future}
%%%%%%%%%%%%%%%%%%%%%%%%%%%%%%%%%%%%%%%%%%%%%%%%%%%%%%%%%%%%%%%%%%%%%%%% 

In summary, this dissertation has pushed the research
on predicative implicit higher-rank polymorphism further, and we believe that
contributions in this dissertation can be used to guide the continued evolution
of (functional) programming language design and implementations. Specifically,
with the new bidirectional type checking algorithm using the \mode mode, we were
able to type-check programs that traditional type inference algorithms cannot,
and thus provide new insights for inference algorithm design with bidirectional
type checking. With the integration of higher-rank polymorphism and gradual
typing, we provided a step forward in gradualizing modern functional programming
languages like Haskell. Moreover, the work on \textit{type promotion} simplified
type inference algorithms with tricky dependency and scoping issues, and the
kind inference for datatypes presented a first known, detailed account of
datatypes, which can serve as a guide for future development of datatypes.

In this section we discuss some future directions we would like to pursue.

\input{Gen/Future/future.lhstex}


%%% Local Variables:
%%% mode: latex
%%% TeX-master: "../Thesis"
%%% org-ref-default-bibliography: ../Thesis.bib
%%% End: