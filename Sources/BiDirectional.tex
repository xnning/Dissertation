%%%%%%%%%%%%%%%%%%%%%%%%%%%%%%%%%%%%%%%%%%%%%%%%%%%%%%%%%%%%%%%%%%%%%%%% 
\chapter{Higher-Rank Polymorphism with Application Mode}
\label{chap:BiDirectional}
%%%%%%%%%%%%%%%%%%%%%%%%%%%%%%%%%%%%%%%%%%%%%%%%%%%%%%%%%%%%%%%%%%%%%%%% 

We have seen in \Cref{sec:DK} that bidirectional type checking is an useful and
versatile tool for type checking and type inference. In traditional
bidirectional type-checking, type information flows from functions to arguments
(e.g., \rref{dk-in-app} in \Cref{sec:DK:declarative}). In this section, we
present a novel variant of bidirectional type checking where the type
information flows from arguments to functions. This variant retains the
inference mode, but adds a so-called \textit{\mode} mode. Such design can remove
annotations that basic bidirectional type checking cannot, and is useful when
type information from arguments is required to type-check the functions being
applied.

We illustrate our novel design of bidirectional type-checking using System \ap,
a lambda calculus with implicit higher-rank polymorphism. This section first
presents the declarative, syntax-directed type system of System \ap
in \Cref{sec:AP:declarative}. The interesting aspects about the new type system
are: 1) the typing rules, which employ a combination of the inference mode and
the \textit{\mode} mode; 2) the novel subtyping relation under an application
context. Later, we prove our type system is type-safe by a type-directed
translation to System F in \Cref{sec:AP:translation}. An algorithmic type system
is discussed in \Cref{sec:AP:algorithm}.



\input{Gen/BiDirectional/introduction}
\input{Gen/BiDirectional/typesystem}
\input{Gen/BiDirectional/translation}
\input{Gen/BiDirectional/algorithm}
\input{Gen/BiDirectional/discussion}



%%% Local Variables:
%%% mode: latex
%%% TeX-master: "../Thesis"
%%% org-ref-default-bibliography: "../Thesis.bib"
%%% End: