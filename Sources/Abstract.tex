%
\noindent

Type inference, as implemented in various modern programming languages,
reconstructs missing types in expressions and increases programmers'
productivity. Modern functional languages such as Haskell come with powerful
forms of type inference. The global type-inference algorithms employed in those
languages are derived from the Hindley-Milner type system, with multiple
extensions. As the languages evolve, researchers also formalize the key aspects
of type inference for the new extensions.

This dissertation studies \textit{predicative implicit higher-rank
  polymorphism}, where polymorphic types can be arbitrarily nested, and
monomorphic types can be inferred automatically. Predicative implicit
higher-rank polymorphism is a common extension that has been studied extensively
in the literature, and has been used pervasively in modern statically typed
programming languages.

The goal of this dissertation is to explore the design space of type inference
for implicit predicative higher-rank polymorphism, as well as to study its
integration with other advanced type system features. The first contribution of
this dissertation is a new type inference algorithm for implicit higher-rank
polymorphism which can accepts programs that many existing type inference
algorithms cannot. The proposed \textit{\mode} mode provides new insights for
\textit{bidirectional type checking}. The second contribution is the first
combination of predicative implicit higher-rank polymorphism with
\textit{gradual typing}, which provides a step forward in gradualizing modern
functional programming languages. The third contribution is an arguably simpler
algorithmic implementation of \textit{subtyping} for higher-rank polymorphism.
The technique developed is then further applied to the \textit{kind inference}
problem for \textit{datatypes}, which provides a first known formulations of
datatype declarations in modern functional programming languages.

% It is our belief that th  the study of predicative
% implicit higher-rank polyrmophism, and the techniques developed in the
% dissertation can be applied to systems beyond those in this dissertations.


% \vspace{1.5\baselineskip}

% \noindent\makebox[\linewidth]{\rule{0.7\textwidth}{0.4pt}}


% \newpage

% \begin{flushright}
%   \null\vspace{\stretch{1}}
%   \textit{To my beloved parents}
%   \vspace{\stretch{2}}\null
% \end{flushright}

%%% Local Variables:
%%% mode: latex
%%% TeX-master: "../Thesis"
%%% org-ref-default-bibliography: "../Thesis.bib"
%%% End: