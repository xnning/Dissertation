%
\noindent
%
% Programs are hard to write. It was so 50 years ago at the time of the so-called
% software crisis; it still remains so nowadays. Over the years, we have
% learned---the hard way---that software should be constructed in a
% \emph{modular} way, i.e., as a network of smaller and loosely connected
% modules. To facilitate writing modular code, researchers and software
% practitioners have developed new methodologies; new programming paradigms;
% stronger type systems; as well as better tooling support. Still, this is not
% enough to cope with today's needs. Several reasons have been raised for the lack
% of satisfactory solutions, but one that is constantly pointed out is the
% inadequacy of existing programming languages for the construction of
% modular software.

% This thesis investigates \emph{disjoint intersection types}, a variant of
% intersection types. Disjoint intersections types have great potential to serve
% as a foundation for powerful, flexible and yet type-safe and easy to reason OO
% languages, suitable for writing modular software. On the theoretical side, this
% thesis shows how to significantly increase the expressiveness of disjoint
% intersection types by adding support for \emph{nested composition}, along with
% \emph{parametric polymorphism}. Nested composition extends inheritance to work
% on a whole family of classes, enabling high degrees of modularity and code
% reuse. The combination with parametric polymorphism further improves the
% state-of-art encodings of extensible designs. However, the extension with nested
% composition and parametric polymorphism is challenging, for two different
% reasons. Firstly, the subtyping relation that supports these features is
% non-trivial. Secondly, the syntactic method used to prove coherence for previous
% calculi with disjoint intersection types is too inflexible. This thesis
% addresses the first problem by adapting and extending the well-known BCD
% subtyping with records, universal quantification and coercions. To address the
% second problem, this thesis proposes a powerful proof method to establish
% coherence. Hence, this thesis puts disjoint intersection types on a solid
% footing by thoroughly exploring their meta-theoretical properties.

% On the pragmatic side, this thesis proposes a new language design with support
% for \emph{first-class traits}, \emph{dynamic inheritance} and nested
% composition. First-class traits allow two objects of statically unknown types
% to be composed without conflicts. Dynamic inheritance allows a class to inherit
% from other classes at \emph{run time}. To address the challenges of typing
% first-class traits and detecting conflicts statically, this thesis shows how to
% model source language constructs for first-class traits and dynamic inheritance
% by leveraging the fine-grained expressiveness of disjoint intersection types. To
% illustrate the applicability of the new design, this thesis conducts a case
% study that modularizes programming language features using a highly modular form
% of \visitor.

% All the results and metatheory presented (unless otherwise indicated) in this
% thesis are mechanized in Coq in order to show the rigorousness of the approach.
% This thesis unifies ideas that are seemingly unrelated but powerful on their
% own---dynamic inheritance, first-class traits, nested composition---by a
% lightweight mechanism, thus providing new insights into software modularity and
% extensibility.

\vspace{1.5\baselineskip}

\noindent\makebox[\linewidth]{\rule{0.7\textwidth}{0.4pt}}


% \newpage

% \begin{flushright}
%   \null\vspace{\stretch{1}}
%   \textit{To my beloved parents}
%   \vspace{\stretch{2}}\null
% \end{flushright}
