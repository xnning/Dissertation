%%%%%%%%%%%%%%%%%%%%%%%%%%%%%%%%%%%%%%%%%%%%%%%%%%%%%%%%%%%%%%%%%%%%%%%% 
\chapter{Background}
\label{chap:Background}
%%%%%%%%%%%%%%%%%%%%%%%%%%%%%%%%%%%%%%%%%%%%%%%%%%%%%%%%%%%%%%%%%%%%%%%% 

This chapter sets the stage for type systems in later chapters. \Cref{sec:HM}
reviews the Hindley-Milner type system \citep{Damas:Milner,hindley,milner}, a
classical type system for the lambda calculus with parametric polymorphism.
\Cref{sec:OL} presents the Odersky-L{\"a}ufer type system
\citep{odersky:putting}, which extends upon the Hindley-Milner type system by
putting higher-rank type annotations to work. Finally in \Cref{sec:DK} we
introduce the Dunfield-Krishnaswami type system, a bidirecitonal higher-rank
type system. Here we pay particular attention to the Dunfield-Krishnaswami
system as it serves as a basis for extensions in later chapters; for example,
\Cref{chap:Gradual} is a direct extension of Dunfield-Krishnaswami. There is
plenty of other related work to higher-rank type system (e.g.,
\cite{practical:inference}), and we include a more substantive discussion of
those work in \Cref{chap:related}.


\input{Gen/Background/HM}
\input{Gen/Background/OL}
\input{Gen/Background/DK}


%%% Local Variables:
%%% mode: latex
%%% TeX-master: "../Thesis"
%%% org-ref-default-bibliography: "../Thesis.bib"
%%% End: