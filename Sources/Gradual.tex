%%%%%%%%%%%%%%%%%%%%%%%%%%%%%%%%%%%%%%%%%%%%%%%%%%%%%%%%%%%%%%%%%%%%%%%% 
\chapter{Gradually Typed Higher-Rank Polymorphism}
\label{chap:Gradual}
%%%%%%%%%%%%%%%%%%%%%%%%%%%%%%%%%%%%%%%%%%%%%%%%%%%%%%%%%%%%%%%%%%%%%%%% 

\textit{Consistent subtyping} is employed in some gradual type systems to validate type
conversions. The original definition by \citet{siek2007gradual} serves as
a guideline for designing gradual type systems with subtyping. Polymorphic
types \`a la System F also induce a subtyping relation that relates
polymorphic types to their instantiations. However
\citeauthor{siek2007gradual}'s definition is not adequate for polymorphic
subtyping.
This section first proposes a generalization of consistent subtyping
(\Cref{sec:gradual:exploration}) that is adequate for polymorphic subtyping, and
subsumes the original definition by \citeauthor{siek2007gradual}. The new
definition of consistent subtyping provides novel insights with respect to
previous polymorphic gradual type systems, which did not employ consistent
subtyping.

We then present \gpc, a gradually typed calculus for implicit
higher-rank polymorphism that uses our new notion of consistent subtyping. We
develop both declarative (\Cref{sec:gradual:type-system}) and bi-directional
algorithmic versions (\Cref{sec:gradual:algorithm}) for the type system. The
algorithmic version employs techniques developed by DK (\Cref{sec:DK}) for
higher-rank polymorphism to deal with instantiation.


\input{Gen/Gradual/introduction}
\input{Gen/Gradual/exploration}
\input{Gen/Gradual/typesystem}
\input{Gen/Gradual/algorithm}
\input{Gen/Gradual/discussion}

%%% Local Variables:
%%% mode: latex
%%% TeX-master: "../Thesis"
%%% org-ref-default-bibliography: "../Thesis.bib"
%%% End: