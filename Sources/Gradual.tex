%%%%%%%%%%%%%%%%%%%%%%%%%%%%%%%%%%%%%%%%%%%%%%%%%%%%%%%%%%%%%%%%%%%%%%%% 
\chapter{Gradually Typed Higher-Rank Polymorphism}
\label{chap:Gradual}
%%%%%%%%%%%%%%%%%%%%%%%%%%%%%%%%%%%%%%%%%%%%%%%%%%%%%%%%%%%%%%%%%%%%%%%% 

Gradual typing~\citep{siek2006gradual} is an increasingly popular topic in both
programming language practice and theory. On the practical side there is a
growing number of programming languages adopting gradual typing. Those languages
include Clojure~\citep{Bonnaire_Sergeant_2016}, Python~\citep{Vitousek_2014, lehtosalo2016mypy},
TypeScript~\citep{typescript}, Hack~\citep{verlaguet2013facebook}, and the
addition of Dynamic to C\#~\citep{Bierman_2010}, to name a few. On the
theoretical side, recent years have seen a large body of research that defines
the foundations of gradual typing~\citep{garcia:abstracting,
  cimini2016gradualizer, CiminiPOPL}, explores their use for both functional and
object-oriented programming~\citep{siek2006gradual, siek2007gradual}, as well as
its applications to many other areas~\citep{Ba_ados_Schwerter_2014,
  castagna2017gradual, Jafery:2017:SUR:3093333.3009865}.

In this chapter, we present \gpc, a gradually typed calculus for implicit
higher-rank polymorphism. Integrating gradual typing with higher-rank
polymorphism is challenging. In particular, gradual typing calculi employ
\textit{type consistency} to validate type conversions. Polymorphic types \`a la
System F also induce a subtyping relation that relates polymorphic types to
their instantiations. The original definition of \textit{consistent subtyping}
by \citet{siek2007gradual} serves as a guideline for designing gradual type
systems with subtyping. However \citeauthor{siek2007gradual}'s definition is not
adequate for polymorphic subtyping. Therefore, this section first proposes a
generalization of consistent subtyping (\Cref{sec:gradual:exploration}) that is
adequate for polymorphic subtyping, and subsumes the original definition by
\citeauthor{siek2007gradual}. The new definition of consistent subtyping
provides novel insights with respect to previous polymorphic gradual type
systems, which did not employ consistent subtyping.

We then develop \gpc on top of our new notion of consistent subtyping. We
develop both declarative (\Cref{sec:gradual:type-system}) and bidirectional
algorithmic versions (\Cref{sec:gradual:algorithm}) for the type system. The
algorithmic version employs techniques developed by DK \citep{DK} for
higher-rank polymorphism to deal with instantiation.


\input{Gen/Gradual/introduction}
\input{Gen/Gradual/exploration}
\input{Gen/Gradual/typesystem}
\input{Gen/Gradual/algorithm}
\input{Gen/Gradual/discussion}

%%% Local Variables:
%%% mode: latex
%%% TeX-master: "../Thesis"
%%% org-ref-default-bibliography: "../Thesis.bib"
%%% End: