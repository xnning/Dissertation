%%%%%%%%%%%%%%%%%%%%%%%%%%%%%%%%%%%%%%%%%%%%%%%%%%%%%%%%%%%%%%%%%%%%%%%%
\chapter{Introduction}
%%%%%%%%%%%%%%%%%%%%%%%%%%%%%%%%%%%%%%%%%%%%%%%%%%%%%%%%%%%%%%%%%%%%%%%%


Modern functional languages such as Haskell, ML, and OCaml come with powerful
forms of type inference. The global type-inference algorithms employed in those
languages are derived from the Hindley-Milner type system (HM)~\citep{hindley,
  Damas:Milner}, with multiple extensions. As the languages evolve, researchers
also formalize the key aspects of type inference for the new extensions. One
common extension of HM, which is also the central theme of this dissertation, is
\emph{higher-rank polymorphism}~\citep{odersky:putting,practical:inference,
  DK}. In particular, we are interested in \textit{predicative implicit
  higher-rank polymorphism}, which extends type inference for functional
programming languages in the presence of polymorphic types.


\input{Gen/Introduction/motivation}
\input{Gen/Introduction/contributions}
\input{Gen/Introduction/organization}


\noindent\makebox[\linewidth]{\rule{0.7\textwidth}{0.4pt}}

\vspace{1.5\baselineskip}



%%% Local Variables:
%%% mode: latex
%%% TeX-master: "../Thesis"
%%% org-ref-default-bibliography: "../Thesis.bib"
%%% End:
