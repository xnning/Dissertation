%%%%%%%%%%%%%%%%%%%%%%%%%%%%%%%%%%%%%%%%%%%%%%%%%%%%%%%%%%%%%%%%%%%%%%%%
\chapter{Introduction}
%%%%%%%%%%%%%%%%%%%%%%%%%%%%%%%%%%%%%%%%%%%%%%%%%%%%%%%%%%%%%%%%%%%%%%%%


\section{Contributions}

In summary the contributions of this thesis are:

\begin{itemize}

\item Section \ref{chap:BiDirectional} proposes a new design for type inference of higher-ranked types.
  \begin{itemize}
  \item We design a variant of bi-directional type checking
    where the inference mode is combined with a new, so-called, application
    mode. The application mode naturally propagates type information from
    arguments to the functions.
  \item With the application mode, we give a new design for type inference of
    higher-ranked type, which generalizes the HM type system, supports a
    polymorphic let as syntactic sugar, and infers higher rank types. We present
    a syntax-directed specification, an elaboration semantics to System F, and
    an algorithmic type system with completeness and soundness proofs.
  \end{itemize}

\item Section \ref{chap:Gradual} presents the type system \gpc, which extends predicative implicit
  higher-rank polymorphism with gradual types.
  \begin{itemize}
  \item We define a framework for consistent subtyping with a new definition of
    consistent subtyping that subsumes and generalizes that of
    \cite{siek:consistent:subtyping} and can deal with polymorphism and top
    types; and a syntax-directed version of consistent subtyping that is sound
    and complete with respect to our definition of consistent subtyping, but
    still guesses instantiations.
  \item Based on consistent subtyping, we present GPC. We prove that our
    calculus satisfies the static aspects of the refined criteria for gradual
    typing \cite{siek:criteria}, and is type-safe by
    a type-directed translation to \pbc \cite{amal:blame}.
  \item We present a sound and complete
    bidirectional algorithm for implementing the declarative system based on
    the design principle of \cite{garcia:principal}.
  \end{itemize}
  
\item We present a new design of the context for ???, where ???.
  \begin{itemize}
  \item We come up with a strategy called promotion that resolves the dependency between types.
  \end{itemize}

\item We further explore the design of promotion by an application to kind
  inference for datatypes.
  \begin{itemize}
  \item We formalize Haskell98’s datatype declarations, providing both a
    declarative specification and syntax-driven algorithm for kind inference. We
    prove that the algorithm is sound and observe how Haskell98’s technique of
    defaulting unconstrained kinds to ⋆ leads to incompleteness. We believe that
    ours is the first formalization of this aspect of Haskell98. Its inclusion
    in this paper both sheds light on this historically important language and
    also prepares us for the more challenging features of modern Haskell.
  \end{itemize}
  


\item A comprehensive Coq mechanization of all metatheory, including type
  safety, coherence, algorithmic soundness and completeness, etc.\footnote{For
    convenience, whenever possible, definitions, lemmas and theorems have hyperlinks (click
    \href{https://github.com/bixuanzju/phd-thesis-artifact}{\leftpointright}) to their Coq counterparts. Also since \fnamee completely
    subsumes \namee, to save work, for \namee metatheory we provide cross
    references to the corresponding \fnamee Coq definitions, instead.} This has
  notably revealed several missing lemmas and oversights in Pierce's manual
  proof of BCD's algorithmic subtyping~\citep{pierce1989decision}. As a
  by-product, we obtain the first mechanically verified BCD-style subtyping
  algorithm with coercions.


\end{itemize}


% The author also contributed to the following publications that do not directly
% relate to the topics of this thesis:
% \begin{itemize}
% \item Ningning Xie, Xuan Bi, Bruno C. d. S. Oliveira. 2018. ``Consistent Subtyping for All''.
%   In \emph{European Symposium on Programming (ESOP)}.
% \item Yanpeng Yang, Xuan Bi, Bruno C. d. S. Oliveira. 2016. ``Unified Syntax with
%   Iso-Types''. In \emph{Asian Symposium on Programming Languages and
%     Systems (APLAS)}.
% \item Tomas Tauber, Xuan Bi, Zhiyuan Shi, Weixin Zhang, Huang Li, Zhenrui Zhang,
%   Bruno C. d. S. Oliveira. 2015. ``Memory-efficient Tail Calls in the JVM with
%   Imperative Functional Objects''. In \emph{Asian Symposium on
%     Programming Languages and Systems (APLAS)}.
% \end{itemize}


\section{Organization}

This thesis is largely based on the
publications by the author~\citep{esop2018:arguments,esop2018:consistent,toplas:consistent,popl:kind},
as indicated below.
\begin{description}
\item[\cref{chap:nested,chap:coherence:simple}:] Ningning Xie and Bruno C. d. S.
  Oliveira. 2018. ``Let Arguments Go First''. In
  \emph{European Symposium on Programming (ESOP)}.
\item[\cref{chap:fi,chap:coherence:poly}:] Ningning Xie, Xuan Bi, and Bruno C. d. S.
  Oliveira. 2018. ``Consistent Subtyping for All''. In
  \emph{European Symposium on Programming (ESOP)}.
\item[\cref{chap:fi,chap:coherence:poly}:] Ningning Xie, Xuan Bi, Bruno C. d. S.
  Oliveira, and Tom Schrijvers. 2019. ``Consistent Subtyping for All''. In
  \emph{ACM Transactions on Programming Languages and Systems (TOPLAS)}.
\item[\cref{chap:traits,chap:case_study}:] Ningning Xie, Richard Eisenber and Bruno C. d. S. Oliveira.
  2020. ``Kind Inference for Datatypes''. In \emph{Kind Inference for Datatypes}.
\end{description}


\noindent\makebox[\linewidth]{\rule{0.7\textwidth}{0.4pt}}

\vspace{1.5\baselineskip}



%%% Local Variables:
%%% mode: latex
%%% TeX-master: "../Thesis"
%%% org-ref-default-bibliography: "../Thesis.bib"
%%% End:
