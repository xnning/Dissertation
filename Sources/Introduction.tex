%%%%%%%%%%%%%%%%%%%%%%%%%%%%%%%%%%%%%%%%%%%%%%%%%%%%%%%%%%%%%%%%%%%%%%%%
\chapter{Introduction}
%%%%%%%%%%%%%%%%%%%%%%%%%%%%%%%%%%%%%%%%%%%%%%%%%%%%%%%%%%%%%%%%%%%%%%%%

mention that in this thesis when we say ``higher-rank polymorphism'' we mean
``predicative implicit higher-rank polymorphism''.

\section{Contributions}

In summary the contributions of this thesis are:


\begin{description}
\item[\Cref{part:typeinference}]
  \begin{itemize}
  \item \Cref{chap:BiDirectional} proposes a new design for type inference of
    higher-rank polymorphism.
    \begin{itemize}
    \item We design a variant of bi-directional type checking
      where the inference mode is combined with a new, so-called, application
      mode. The application mode naturally propagates type information from
      arguments to the functions.
    \item With the application mode, we give a new design for type inference of
      higher-rank polymorphism, which generalizes the HM type system, supports a
      polymorphic let as syntactic sugar, and infers higher rank types. We
      present a syntax-directed specification, an elaboration semantics to
      System F, and an algorithmic type system with completeness and soundness
      proofs.
    \end{itemize}

  \item \Cref{chap:Promotion} presents a new approach for implementing unification.
    \begin{itemize}
    \item We propose a process named \textit{promotion}, which, given a
      unification variable and a type, promotes the type so
      that all unification variables in the type are well-typed with regard to
      the unification variable.
    \item We apply promotion in a new implementation of the unification
      procedure in higher-rank polymorphism, and show that the new
      implementation is sound and complete.
    \end{itemize}
  \end{itemize}

\item[\Cref{part:extensions}]
  \begin{itemize}

  \item \cref{chap:Gradual} extends higher-rank
    polymorphism with gradual types.
    \begin{itemize}
    \item We define a framework for consistent subtyping with
      \begin{itemize}
      \item a new definition of
        consistent subtyping that subsumes and generalizes that of
        \cite{siek:consistent:subtyping} and can deal with polymorphism and top
        types;
      \item and a syntax-directed version of consistent subtyping that is sound
        and complete with respect to our definition of consistent subtyping, but
        still guesses instantiations.
      \end{itemize}
    \item Based on consistent subtyping, we present he calculus GPC. We prove that
      our calculus satisfies the static aspects of the refined criteria for
      gradual typing \citep{siek:criteria}, and is type-safe by a type-directed
      translation to \pbc \citep{amal:blame}.
    \item We present a sound and complete
      bidirectional algorithm for implementing the declarative system based on
      the design principle of \cite{garcia:principal}.
    \end{itemize}
  \item \Cref{chap:kindinference} further explores the design of promotion in the context of kind
    inference for datatypes.
    \begin{itemize}
    \item We formalize Haskell98’s datatype declarations, providing both a
      declarative specification and syntax-driven algorithm for kind inference. We
      prove that the algorithm is sound and observe how Haskell98’s technique of
      defaulting unconstrained kinds to $\star$ leads to incompleteness. We
      believe that ours is the first formalization of this aspect of Haskell98.
      % Its inclusion in this work both sheds light on this historically important
      % language and also prepares us for the more challenging features of modern
      % Haskell.
    \item We then present a type and kind language that is unified and dependently
      typed, modeling the challenging features for kind inference in modern
      Haskell. We include both a declarative specification and a syntax-driven
      algorithm. The algorithm is proved sound, and we observe where and why
      completeness fails. In the design of our algorithm, we must choose between
      completeness and termination; we favor termination but conjecture that an
      alternative design would regain completeness. Unlike other dependently typed
      languages, we retain the ability to infer top-level kinds instead of relying
      on compulsory annotations.
    \end{itemize}
  \end{itemize}
\end{description}

Many metatheory in the paper comes with Coq proofs, including type safety,
coherence, etc.\footnote{For convenience, whenever possible, definitions, lemmas
  and theorems have hyperlinks (click \leftpointright) to their Coq
  counterparts. }

\section{Organization}

This thesis is largely based on the
publications by the author~\citep{esop2018:arguments,esop2018:consistent,toplas:consistent,popl:kind,tfp},
as indicated below.
\begin{description}
\item[\cref{chap:BiDirectional}:] Ningning Xie and Bruno C. d. S.
  Oliveira. 2018. ``Let Arguments Go First''. In
  \emph{European Symposium on Programming (ESOP)}.
\item[\cref{chap:Gradual}:] Ningning Xie, Xuan Bi, and Bruno C. d. S.
  Oliveira. 2018. ``Consistent Subtyping for All''. In
  \emph{European Symposium on Programming (ESOP)}.
\item[\cref{chap:Dynamic}] Ningning Xie, Xuan Bi, Bruno C. d. S.
  Oliveira, and Tom Schrijvers. 2019. ``Consistent Subtyping for All''. In
  \emph{ACM Transactions on Programming Languages and Systems (TOPLAS)}.
\item[\cref{chap:Promotion}:]
  Ningning Xie and Bruno C. d. S. Oliveira.
  2017. ``Towards Unification for Dependent Types'' (Extended abstract), In \emph{Draft Proceedings
    of Trends in Functional Programming (TFP)}.
\item[\cref{chap:kindinference}:]
  Ningning Xie, Richard Eisenberg and Bruno C. d. S. Oliveira. 2020. ``Kind
  Inference for Datatypes''. In \emph{Symposium on Principles of Programming
    Languages (POPL)}.
\end{description}


\noindent\makebox[\linewidth]{\rule{0.7\textwidth}{0.4pt}}

\vspace{1.5\baselineskip}



%%% Local Variables:
%%% mode: latex
%%% TeX-master: "../Thesis"
%%% org-ref-default-bibliography: "../Thesis.bib"
%%% End:
